\documentclass{article}

\usepackage{verse}

\title{Divina Commedia}

\author{Dante Alighieri}

\date{25 marzo 1300}

\begin{document}

\maketitle

\large

\poemtitle{Inferno}
\settowidth{\versewidth}{Nel mezzo del cammin di nostra vita}

\begin{verse}[\versewidth]
  \poemlines{3}
  \indentpattern{100}
  \flagverse{I}
  \begin{patverse*}
    \setverselinenums{1}{1}
    Nel mezzo del cammin di nostra vita\\
    mi ritrovai per una selva oscura,\\
    ché la diritta via era smarrita.\\
    Ahi quanto a dir qual era è cosa dura\\
    esta selva selvaggia e aspra e forte\\
    che nel pensier rinova la paura!\\
    Tant'è amara che poco è più morte;\\
    ma per trattar del ben ch'i' vi trovai,\\
    dirò de l'altre cose ch'i' v'ho scorte.\\
    Io non so ben ridir com'i' v'intrai,\\
    tant'era pien di sonno a quel punto\\
    che la verace via abbandonai.
  \end{patverse*}
\end{verse}

\begin{verse}[\versewidth]
  \poemlines{3}
  \indentpattern{100}
  \flagverse{III}
  \begin{patverse*}
    \setverselinenums{1}{1}
    ``Per me si va ne la città dolente,\\
    per me si va ne l'etterno dolore,\\
    per me si va tra la perduta gente.\\
    \emph{Giustizia mosse il mio alto fattore\\
    fecemi la divina podesate,\\
    la somma sapienza e 'l primo amore.\\
    Dinanzi a me non fuour cose create\\
    se non etterne, e io etterno duro.}\\
    Lasciate ogne speranza, voi ch'intrate.''\\
    \setverselinenums{22}{1}
    Quivi sospiri, pianti e alti guai\\
    risonavan per l'aere sanza stelle,\\
    per ch'io al cominciar ne lagrimai.\\
    Diverse lingue, orribili favelle,\\
    parole di dolore, accenti d'ira,\\
    voci alte e fioche, e suon di man con elle\\
    facevano un tumulto, il qual s'aggira\\
    sempre in quell'aura sanza tempo tinta,\\
    come la rena quando turbo spira.
  \end{patverse*}
\end{verse}

\begin{verse}[\versewidth]
  \poemlines{3}
  \indentpattern{100}
  \flagverse{XII}
  \begin{patverse*}
    \setverselinenums{100}{1}
    Or ci movemmo con la scorta fida\\
    lungo la proda del bollor vermiglio,\\
    dove i bolliti facieno alte strida.
  \end{patverse*}
\end{verse}

\begin{verse}[\versewidth]
  \poemlines{3}
  \indentpattern{100}
  \flagverse{XIII}
  \begin{patverse*}
    \setverselinenums{1}{1}
    Non era ancor di là Nesso arrivato,\\
    quando noi ci mettemmo per un bosco\\
    che da neun sentiero era segnato.\\
    Non fronda verde, ma di color fosco;\\
    non rami schietti, ma nodosi e 'nvolti;\\
    non pomi v'eran, ma stecchi con tòsco.\\
    \emph{Non han sí aspri sterpi né sí folti \\
    quelle fiere selvagge che 'n odio hanno\\
    tra Cecina e Corneto i luoghi colti.}\\
    Quivi le brutte Arpie lor nidi fanno,\\
    \emph{che cacciar de le Strofade i Troiani\\
    con tristo annunzio di futuro danno.}\\
    Ali hanno late, e colli e visi umani,\\
    piè con artigli, e pennuto 'l gran ventre;\\
    fanno lamenti in su li alberi strani.\\
    \setverselinenums{22}{1}
    Io sentia d'ogne parte trarre guai,\\
    e non vedea persona che 'l facesse;\\
    per ch'io tutto smarrito m'arrestai. \\
    \setverselinenums{28}{1}
    \emph{Però disse 'l maestro:} ``Se tu tronchi\\
    qualche fraschetta d'una d'este piante,\\
    li pensier c'hai si faran tutti monchi''\\
    \emph{Allor porsi la mano un poco avante\\
    e colsi un ramicel da un gran pruno;\\
    e 'l tronco suo gridò:} ``Perché mi schiante?''.\\
    \emph{Da che fatto fu poi di sangue bruno,\\
    ricominciò a dir:} ``Perché mi scerpi?\\
    non hai tu spirto di pietade alcuno?\\
    Uomini fummo, e or siam fatti sterpi:\\
    ben dovrebb'esser la tua man più pia,\\
    se state fossimo anime di serpi''.\\
    \setverselinenums{58}{1}
    Io son colui che tenni ambo le chiavi\\
    del cor di Federigo, e che le volsi,\\
    serrando e diserrando, sí soavi,\\
    \setverselinenums{70}{1}
    L'animo mio, per disdegnoso gusto,\\
    credendo col morir fuggir disdegno,\\
    ingiusto fece me contra me giusto.
  \end{patverse*}
\end{verse}

\begin{verse}[\versewidth]
  \poemlines{3}
  \indentpattern{100}
  \flagverse{XXXIV}
  \begin{patverse*}
    \emph{``Vexilla regis prodeunt inferni\\
    verso di noi; però dinanzi mira'',\\
    disse 'l maestro mio, ``se tu 'l discerni''.}\\
    Come quando una grossa nebbia spira,\\
    o quando l'emisperio nostro annotta,\\
    par di lungi un molin che 'l vento gira,\\
    veder mi parve un tal dificio allotta;\\
    poi per lo vento mi ristrinsi retro\\
    al duca mio; ché non lì era altra grotta.\\
    \setverselinenums{20}{2}
    ``Ecco Dite'', \emph{dicendo}, ``ed ecco il loco\\
    ove convien che di fortezza t'armi''.\\
    Com'io divenni allor gelato e fioco,\\
    nol dimandar, lettor, ch'i' non lo scrivo,\\
    \emph{però ch'ogne parlar sarebbe poco.}\\
    Io non mori' e non rimasi vivo:\\
    \emph{pensa oggimai per te, s'hai fior d'ingegno,\\
    qual io divenni, d'uno e d'altro privo.}\\
    Lo 'mperador del doloroso regno\\
    da mezzo 'l petto uscìa fuor de la ghiaccia;\\
    e più con un gigante io mi convegno,\\
    \setverselinenums{37}{1}
    Oh quanto parve a me gran maraviglia\\
    quand'io vidi tre facce a la sua testa!\\
    \setverselinenums{46}{1}
    Sotto ciascuna uscivan due grand'ali,\\
    quanto si convenia a tanto uccello:\\
    vele di mar non vid'io mai cotali.\\
    Non avean penne, ma di vispistrello\\
    era lor modo; e quelle svolazzava,\\
    sí che tre venti si movean da ello:\\
    quindi Cocito tutto s'aggelava.\\
    Con sei occhi piangea, e per tre menti\\
    gocciava 'l pianto e sanguinosa bava.\\
    \setverselinenums{133}{1}
    Lo duca e io per quel cammino ascoso\\
    intrammo a ritornar nel chiaro mondo;\\
    e sanza cura aver d'alcun riposo,\\
    salimmo sù, el primo e io secondo,\\
    tanto ch'i' vidi de le cose belle\\
    che porta 'l ciel, per un pertugio tondo.\\
    E quindi uscimmo a riveder le stelle.
  \end{patverse*}
\end{verse}

\poemtitle{Purgatorio}
\settowidth{\versewidth}{There was an old party of Lyme}

\begin{verse}[\versewidth]
  \poemlines{3}
  \indentpattern{100}
  \flagverse{I}
  \begin{patverse*}
    \setverselinenums{1}{1}
    Per correr miglior acque alza le vele\\
    omai la navicella del mio ingegno,\\
    che lascia dietro a sé mar sí crudele;\\
    e canterò di quel secondo regno\\
    dove l'umano spirito si purga\\
    e di salire al ciel diventa degno.\\
    \setverselinenums{13}{1}
    Dolce color d'oriental zaffiro,\\
    che s'accoglieva nel sereno aspetto\\
    del mezzo, puro infino al primo giro,\\
    a li occhi miei ricominciò diletto,\\
    tosto ch'io usci' fuor de l'aura morta\\
    che m'avea contristati li occhi e 'l petto.\\
    \setverselinenums{115}{1}
    L'alba vinceva l'ora mattutina\\
    che fuggia innanzi, sí che di lontano\\
    conobbi il tremolar de la marina.\\
    Noi andavam per lo solingo piano\\
    com'om che torna a la perduta strada,\\
    che 'nfino ad essa li pare ire in vano.\\
    \setverselinenums{130}{1}
    Venimmo poi in sul lito diserto,\\
    che mai non vide navicar sue acque\\
    omo, che di tornar sia poscia esperto.\\
  \end{patverse*}
\end{verse}

\begin{verse}[\versewidth]
  \poemlines{3}
  \indentpattern{100}
  \flagverse{IX}
  \begin{patverse*}
    \setverselinenums{49}{1}
    Tu se' omai al purgatorio giunto:\\
    \setverselinenums{76}{1}
    vidi una porta, e tre gradi di sotto\\
    per gire ad essa, di color diversi,\\
    e un portier ch'ancor non facea motto.\\
    \emph{E come l'occhio più e più v'apersi,\\
    vidil seder sovra 'l grado sovrano,\\
    tal ne la faccia ch'io non lo soffersi;}\\
    e una spada nuda avea in mano,\\
    che reflettea i raggi sí ver' noi,\\
    ch'io drizzava spesso il viso in vano.\\
    \setverselinenums{112}{1}
    Sette P ne la fronte mi descrisse\\
    col punton de la spada, e ``Fa che lavi,\\
    quando se' dentro, queste piaghe'' disse.\\
    \setverselinenums{130}{1}
    Poi pinse l'uscio a la porta sacrata,\\
    dicendo: ``Intrate; ma facciovi accorti\\
    che di fuor torna chi 'n dietro si guata''.\\
  \end{patverse*}
\end{verse}

\begin{verse}[\versewidth]
  \poemlines{3}
  \indentpattern{100}
  \flagverse{X}
  \begin{patverse*}
    \setverselinenums{112}{1}
    \emph{o cominciai}: ``Maestro, quel ch'io veggio\\
    muovere a noi, non mi sembian persone,\\
    e non so che, sí nel veder vaneggio''.\\
    Ed elli a me: ``La grave condizione\\
    di lor tormento a terra li rannicchia,\\
    sí che ' miei occhi pria n'ebber tencione.\\
    Ma guarda fiso là, e disviticchia\\
    col viso quel che vien sotto a quei sassi:\\
    già scorger puoi come ciascun si picchia''.\\
    O superbi cristian, miseri lassi,\\
    che, de la vista de la mente infermi,\\
    fidanza avete ne' retrosi passi,\\
    non v'accorgete voi che noi siam vermi\\
    nati a formar l'angelica farfalla,\\
    che vola a la giustizia sanza schermi?
    \end{patverse*}
\end{verse}

\begin{verse}[\versewidth]
  \poemlines{3}
  \indentpattern{100}
  \flagverse{XI}
  \begin{patverse*}
    \setverselinenums{112}{1}
    Oh vana gloria de l'umane posse!\\
    com' poco verde in su la cima dura,\\
    se non è giunta da l'etati grosse!\\
    \emph{Credette Cimabue ne la pittura\\
    tener lo campo, e ora ha Giotto il grido,\\
    sí che la fama di colui è scura.\\
    Così ha tolto l'uno a l'altro Guido\\
    la gloria de la lingua; e forse è nato\\
    chi l'uno e l'altro caccerà del nido.}\\
    Non è il mondan romore altro ch'un fiato\\
    di vento, ch'or vien quinci e or vien quindi,\\
    e muta nome perché muta lato.
  \end{patverse*}
\end{verse}

\begin{verse}[\versewidth]
  \poemlines{3}
  \indentpattern{100}
  \flagverse{XII}
  \begin{patverse*}
    \setverselinenums{13}{1}
    \emph{ed el mi disse}: ``Volgi li occhi in giùe:\\
    buon ti sarà, per tranquillar la via,\\
    veder lo letto de le piante tue''.
  \end{patverse*}
\end{verse}

\begin{verse}[\versewidth]
  \poemlines{3}
  \indentpattern{100}
  \flagverse{X}
  \begin{patverse*}
    \setverselinenums{94}{1}
    Colui che mai non vide cosa nova\\
    produsse esto visibile parlare,\\
    novello a noi perché qui non si trova.
  \end{patverse*}
\end{verse}

\begin{verse}[\versewidth]
  \poemlines{3}
  \indentpattern{100}
  \flagverse{XII}
  \begin{patverse*}
    \setverselinenums{25}{1}
    Vedea colui che fu nobil creato\\
    più ch'altra creatura, giù dal cielo\\
    folgoreggiando scender, da l'un lato.\\
    \setverselinenums{43}{1}
    O folle Aragne, sí vedea io te\\
    già mezza ragna, trista in su li stracci\\
    de l'opera che mal per te si fé.\\
    \setverselinenums{61}{1}
    Vedeva Troia in cenere e in caverne;\\
    o Ilión, come te basso e vile\\
    mostrava il segno che lì si discerne!\\
  \end{patverse*}
\end{verse}

\begin{verse}[\versewidth]
  \poemlines{3}
  \indentpattern{100}
  \flagverse{XIII}
  \begin{patverse*}
    \setverselinenums{1}{1}
    Noi eravamo al sommo de la scala,\\
    dove secondamente si risega\\
    lo monte che salendo altrui dismala.\\
    \setverselinenums{37}{1}
    \emph{E 'l buon maestro}: ``Questo cinghio sferza\\
    la colpa de la invidia, e però sono\\
    tratte d'amor le corde de la ferza.\\
    \setverselinenums{52}{1}
    Non credo che per terra vada ancoi\\
    omo sí duro, che non fosse punto\\
    per compassion di quel ch'i' vidi poi;\\
    ché, quando fui sí presso di lor giunto,\\
    che li atti loro a me venivan certi,\\
    per li occhi fui di grave dolor munto.\\
    Di vil ciliccio mi parean coperti,\\
    e l'un sofferia l'altro con la spalla,\\
    e tutti da la ripa eran sofferti.\\
    \emph{Così li ciechi a cui la roba falla,\\
    stanno a' perdoni a chieder lor bisogna,\\
    e l'uno il capo sopra l'altro avvalla,\\
    perché 'n altrui pietà tosto si pogna,\\
    non pur per lo sonar de le parole,\\
    ma per la vista che non meno agogna.}\\
    E come a li orbi non approda il sole,\\
    così a l'ombre quivi, ond'io parlo ora,\\
    luce del ciel di sé largir non vole;\\
    ché a tutti un fil di ferro i cigli fóra\\
    e cusce sí, come a sparvier selvaggio\\
    si fa però che queto non dimora.
  \end{patverse*}
\end{verse}

\begin{verse}[\versewidth]
  \poemlines{3}
  \indentpattern{100}
  \flagverse{XXV}
  \begin{patverse*}
    \setverselinenums{109}{1}
    E già venuto a l'ultima tortura\\
    s'era per noi, e vòlto a la man destra,\\
    ed eravamo attenti ad altra cura.\\
    \setverselinenums{124}{1}
    e vidi spirti per la fiamma andando;\\
    per ch'io guardava a loro e a' miei passi,\\
    compartendo la vista a quando a quando.
  \end{patverse*}
\end{verse}

\begin{verse}[\versewidth]
  \poemlines{3}
  \indentpattern{100}
  \flagverse{XXVI}
  \begin{patverse*}
    \setverselinenums{82}{1}
    Nostro peccato fu ermafrodito;\\
    ma perché non servammo umana legge,\\
    seguendo come bestie l'appetito,\\
    in obbrobrio di noi, per noi si legge,\\
    quando partinci, il nome di colei\\
    che s'imbestiò ne le 'mbestiate schegge.
  \end{patverse*}
\end{verse}

\begin{verse}[\versewidth]
  \poemlines{3}
  \indentpattern{100}
  \flagverse{XXVIII}
  \begin{patverse*}
    \setverselinenums{1}{1}
    Vago già di cercar dentro e dintorno\\
    la divina foresta spessa e viva,\\
    ch'a li occhi temperava il novo giorno,\\
    sanza più aspettar, lasciai la riva,\\
    prendendo la campagna lento lento\\
    su per lo suol che d'ogne parte auliva.\\
    Un'aura dolce, sanza mutamento\\
    avere in sé, mi feria per la fronte\\
    non di più colpo che soave vento;
  \end{patverse*}
\end{verse}

\begin{verse}[\versewidth]
  \poemlines{3}
  \indentpattern{100}
  \flagverse{XXX}
  \begin{patverse*}
    \setverselinenums{31}{1}
    sovra candido vel cinta d'uliva\\
    donna m'apparve, sotto verde manto\\
    vestita di color di fiamma viva.\\
    \emph{E lo spirito mio, che già cotanto\\
    tempo era stato ch'a la sua presenza\\
    non era di stupor, tremando, affranto,\\
    sanza de li occhi aver più conoscenza,\\
    per occulta virtù che da lei mosse,\\
    d'antico amor sentì la gran potenza.\\
    Tosto che ne la vista mi percosse\\
    l'alta virtù che già m'avea trafitto\\
    prima ch'io fuor di puerizia fosse,\\
    volsimi a la sinistra col respitto\\
    col quale il fantolin corre a la mamma\\
    quando ha paura o quando elli è afflitto,\\
    per dicere a Virgilio}: ``Men che dramma\\
    di sangue m'è rimaso che non tremi:\\
    conosco i segni de l'antica fiamma''.
  \end{patverse*}
\end{verse}

\begin{verse}[\versewidth]
  \poemlines{3}
  \indentpattern{100}
  \flagverse{XXXIII}
  \begin{patverse*}
    \setverselinenums{136}{1}
    \emph{S'io avessi, lettor, più lungo spazio\\
    da scrivere, i' pur cantere' in parte\\
    lo dolce ber che mai non m'avria sazio;\\
    ma perché piene son tutte le carte\\
    ordite a questa cantica seconda,\\
    non mi lascia più ir lo fren de l'arte.}\\
    Io ritornai da la santissima onda\\
    rifatto sí come piante novelle\\
    rinovellate di novella fronda,\\
    puro e disposto a salire a le stelle.
  \end{patverse*}
\end{verse}

\end{document}
